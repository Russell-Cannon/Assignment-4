\documentclass[12pt]{article}
\usepackage[margin=1in]{geometry}
\usepackage{setspace}
\usepackage{harvard}
\usepackage{listings}
\usepackage{multicol}
\bibliographystyle{apsr}
\usepackage{algorithm}
\usepackage{algpseudocode}
\usepackage{amsmath}
\usepackage{graphicx} 
\usepackage{forest}
\usepackage{hyperref}

\begin{document}
\title{Theory and Code Task 4}
\author{Russell Cannon, Ian Mooney, Patrick Murphy}

\maketitle
\singlespacing

\begin{abstract}
\begin{center}
citations
\end{center}
\end{abstract}

\newpage

\section{Key Reports}
\begin{enumerate}
\item 
The occupancy ratio to be used in linear probing. This involves experimenting with different values such
as 50\%, 70\%, and 80\%, and reporting runtimes in nanoseconds.

\item
Optimizing chain length in open hashing. At least three experiments should be conducted, and runtimes
in nanoseconds should be reported.

\item
Experimentation with different hash functions. A simple function such as f (r) = r\%hsize should be the
initial attempt.

\item
Handling collisions in the table for linear probing. The collision resolution method implemented must be
described, with research and inclusion of a method described in the lecture.

\item
The necessity of an interface file (a “.h” file) for the functions implemented.

\item
Writing a function to prompt a user for a word, display the number of occurrences of this word in the
text, and the locations of said occurrences in “The Adventure of the Engineer’s Thumb”.

\item
Implementing a function to output a list of the 80 least frequently occurring words in the text.

\item
Implementing a function to output a list of the 80 most frequently occurring words in the text.
\end{enumerate}

\end{document}